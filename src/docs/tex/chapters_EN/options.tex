\section{Options}
The \emph{Options} menu can always be accessed by pressing \keybinding{ESC} until you reach the main screen $\rightarrow$ Options.

\emph{NB}: please submit patches to fill in the empty sections, the authors of this text are not at all conversant with the ins and outs of video displays.

\subsection{Video}
This section offers you various ways to make UFO:AI look the best way possible to the engine and your system. Please be aware that while most options here can cause improved graphics, they can also slow down your computer remarkably.

\subsubsection*{Resolution}
You may choose resolutions ranging from 320x240 to 2048x1536. It might be worth noting that after the `standard' resolutions, some rather rare resolutions like 1280x854 and the like follow, which might be interesting for laptop users. You can also set custom resolutions if you set the cvar \cvar{vid\_mode} to \cvarvalue{-1} and use the cvars \cvar{vid\_width} and \cvar{vid\_height} to define your desired resolution.

\subsubsection*{Fullscreen}
Enable or disable fullscreen options.

\subsubsection*{Texture compression}
\subsubsection*{Texture resolution cap}
\subsubsection*{Show FPS}
If you choose to turn on this option UFO:AI will display current frames per second in the very upper right corner.
\subsubsection*{Texture anisotropy level}
\subsubsection*{Texture Lod}
\subsubsection*{Image filter}
\subsubsection*{Gamma}
Here you may adjust the Gamma factor to your graphic card or monitor settings.  Note that on some platforms (MacOS) this may affect your whole environment.  Gamma affects the `brightness' of your display.

\subsection{Sound}

\emph{NB}: please submit patches to fill in the empty sections, the authors of this text are not at all conversant with the ins and outs of audio either.

\subsubsection*{Effects}
Use this fader to adjust effects volume for your neighbours' ears.

\subsubsection*{Music}
Use this fader to adjust music volume once you get bored of your private music collection.

\subsubsection*{Mixing rate}
Increase the quality of the audio.

\subsection{Game}
Besides having the chance to change your Player Name, the game options also offer more practical opportunities.  Some of these affect only new games, while others cause changes in games that are currently running.

\subsubsection*{Start with employees}
Choosing this option will make you start with a set of employees as well as some basic equipment for your soldiers. If you prefer to do really everything on your own, switch to ``no'' here.

\subsubsection*{Start with buildings}
If you say ``yes'' here UFO:AI will equip your first base with standard set of facilities that should do the trick quite nicely. Perfectionists may wish to choose ``no'' here.

\subsubsection*{Confirm actions}
You may want to enable this option in order to prevent mistakes or to make it easier to play UFO:AI while drunk. Doing so will make Battlescape show you the path your soldiers will choose once ordered to move to a certain spot. In order to finally make the soldier in question move there, you will need to press \keybinding{Enter}.

\subsubsection*{HUD design}
This toggles between the classic and altHUD for the tactical combat interface. While you can change this option during combat, it will not take effect until your next battle.

\subsubsection*{Center view}
This influences whether or not the HUD will focus on the selected soldier as you change between members of your squad.

\subsubsection*{Cursor tooltips}
Turn on/off cursor tooltips, indicating the function of the various UI elements.

\subsubsection*{Camera scroll}
Adjust how quickly the camera will scroll.

\subsubsection*{Camera rotation}
Adjust how quickly the camera will rotate.
